\documentclass[11pt]{article}
\usepackage{fullpage}
\usepackage{amsmath}
\usepackage{amssymb}
\usepackage{graphicx}

\begin{document}


\title{RTLgen - Functional Specification Document}
\author{Nitij Mangal}
\date{January 2009}
\maketitle

%%%%%%%%%%%%%%%%%%%%%%%%%%%%%%%%%%%%%%%%%%%%%%%%%%%%%%%%%%%%%%%%%%%%%%%%%%%%%%%%
\section{Overview}
\textbf{RTLgen} is an EDA tool which provides a perl based development environment to generate RTL. It currently support verilog and will be extended to VHDL over time. The tool is an aid to the designer as it off loads all syntax and language specific tasks so that the designer can focus on the core functionality and logic of the design. RTLgen can currently perform the following tasks
\begin{itemize}
	\item Create port lists and declarations
	\item Create net declarations
	\item Create sensitivity lists for combinational and sequential always blocks
	\item Create instance declarations and connections
\end{itemize}

%%%%%%%%%%%%%%%%%%%%%%%%%%%%%%%%%%%%%%%%%%%%%%%%%%%%%%%%%%%%%%%%%%%%%%%%%%%%%%%%
\section{Tool flow}
RTLgen allows a designer to call high level construct in a pseudo RTL (RTL like language) which are processed by the tool to output actual RTL code.The tool will work in three stages
\begin{enumerate}
	\item Stage1: Parse the pesudo RTL
	\item Stage2: Analyze the design
	\item Stage3: Output full blown RTL
\end{enumerate}

\begin{enumerate}
\item \textbf{Stage1: Parse the pesudo RTL} \\
The tool uses a Verilog parser built on top of Hardware::Verilog and Parse::RecDescent packages from CPAN. The verilog parser has been enhanced to parse the following pseudo RTL constructs.
	\begin{enumerate}
	\item ::Module \emph{moduleName}; 
	\item ::Clk \emph{clkList};
	\item ::Reset \emph{rstList};
	\item ::Ports;
	\item ::Nets;
	\item ::Wires;
	\item ::Regs;
	\item ::Always \emph{seq -clk 'clk' -rst 'rst'|comb};
	\item ::Instance \emph{-f filename -m module instanceName}; \\ ::s/string1/string2/;
	\end{enumerate}
The tool will store the following in a internal datastructure. 
	\begin{enumerate}
	\item module name
	\item declared ports
	\item declared nets - wires and regs
	\item instances and already connected ports
	\item always blocks and sensitivity lists
	\item all nets used as lvalues (count and context)
	\item all nets used on right side of expressions (count and context)
	\end{enumerate}
\textbf{Casey please add details of DB here}\\

\item Stage2: Analyze the design
	In this stage RTLgen will do the following tasks
	\begin{enumerate}
	\item For each net, count number of source and sinks and the context of source or sink and cataegorize each net as input, output, wire or reg. Compare against the already declared ports, wires and regs, if any and check the type and width are declared correctly. Keep track of undeclared ports, wires and regs.
	\item For each always block, create sensitivity list
	\item For each instance, read the file which has the module defined, recursively call RTLgen if it is also \textbf{.ehv} file, and keep track of portlist of the instance and regexp replace for connectivity.
	\end{enumerate}
\item Stage3: Output full blown RTL
	\begin{enumerate}
	\item \textbf{::Module \emph{moduleName}; }\\
		expand portlist
	\item \textbf{::Clk \emph{clkList};}\\
	\item \textbf{::Reset \emph{rstList};}\\
	\item \textbf{::Ports;}\\
		expand input, output and inout declarations
	\item \textbf{::Nets;}\\
		wires and regs declarations
	\item \textbf{::Wires:}\\
		wires declarations
	\item \textbf{::Regs:}\\
		regs declarations
	\item \textbf{::Always \emph{seq -clk 'clk' -rst 'rst'|comb}:}\\
		expand sensitivity list, use default clock and reset if not specified for a seq block
	\item \textbf{::Instance \emph{-f filename -m module instanceName}:}\\
		expand instance connectivity
	\end{enumerate}
\end{enumerate}

%%%%%%%%%%%%%%%%%%%%%%%%%%%%%%%%%%%%%%%%%%%%%%%%%%%%%%%%%%%%%%%%%%%%%%%%%%%%%%%%
\section{Detailed description of supported funtionality}
\begin{itemize}
\item put here
\end{itemize}

%%%%%%%%%%%%%%%%%%%%%%%%%%%%%%%%%%%%%%%%%%%%%%%%%%%%%%%%%%%%%%%%%%%%%%%%%%%%%%%%
\section{RTLgen pseudo RTL language}

%%%%%%%%%%%%%%%%%%%%%%%%%%%%%%%%%%%%%%%%%%%%%%%%%%%%%%%%%%%%%%%%%%%%%%%%%%%%%%%%
\section{Examples}


%%%%%%%%%%%%%%%%%%%%%%%%%%%%%%%%%%%%%%%%%%%%%%%%%%%%%%%%%%%%%%%%%%%%%%%%%%%%%%%%
\section{Next Steps}
\begin{enumerate}
\item add support for VHDL
\end{enumerate}

 
\end{document}
